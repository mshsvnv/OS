\chapter{Функции обработчика прерывания от системного таймера}

\section{Windows}

\subsection*{По тику}
\begin{itemize}
	\item инкремент счетчика системного времени;
	\item декремент остатка кванта текущего потока;
	\item декремент счетчиков отложенных задач;
	\item если активен механизм профилирования ядра, то инициализация отложенного вызова обработчика ловушки профилирования ядра добавлением объекта в очередь \textit{DPC} (Deferred procedure call, отложенный вызов процедуры). Обработчик ловушки профилирования регистрирует адрес команды, выполнявшейся на момент прерывания. 
\end{itemize}

\subsection*{По главному тику}
\begin{itemize}
	\item инициализация диспетчера настройки баланса путем освобождения объекта <<событие>>, на котором он ожидает.
\end{itemize}

\subsection*{По кванту}
\begin{itemize}
	\item инициация диспетчеризации потоков добавлением соответствующего объекта в очередь \textit{DPC}.
\end{itemize}

\section{UNIX/Linux}

% 187 вахалия

\subsection*{По тику}
\begin{itemize}
	\item инкремент счетчика тиков аппаратного таймера;
	\item инкремент часов и других таймеров системы;
	\item обновление статистики использования процессора текущим процессом (инкрементмент поля \texttt{p\_cpu} дескриптора текущего процесса до максимального значения, равного 127);
	\item декремент счетчика времени, оставшегося до отправления на выполнение отложенных вызовов, при достижении нулевого значения счетчика -- выставление флага, указывающего на необходимость запуска обработчика отложенного вызова;
	\item декремент кванта текущего потока.
\end{itemize}

\subsection*{По главному тику}
\begin{itemize}
	\item регистрация отложенных вызовов функций, относящихся к работе планировщика,
	таких как пересчет приоритетов;
	\item пробуждение (то есть регистрирация отложенного вызова процедуры \texttt{wakeup}, которая перемещает дескриптор процесса из списка <<спящих>> в очередь <<готовых к выполнению>>) системных процессов \texttt{swapper} и \texttt{pagedaemon};
	\item декремент счетчика времени, оставшегося до посылки одного из следующих сигналов:
	\begin{itemize}
		\item \texttt{SIGALRM}~--- сигнал, посылаемый процессу по истечении времени, заданного функцией \texttt{alarm()} (будильник реального времени);
		\item \texttt{SIGPROF}~--- сигнал, посылаемый процессу по истечении времени, заданного в таймере профилирования (будильник профиля процесса);
		\item \texttt{SIGVTALRM}~--- сигнал, посылаемый процессу по истечении времени работы в режиме задачи (будильник виртуального времени).
	\end{itemize}
\end{itemize}

\subsection*{По кванту}
\begin{itemize}
	\item если текущий процесс превысил выделенный ему квант процессорного времени, отправка ему сигнала \texttt{SIGXCPU}.
\end{itemize}



